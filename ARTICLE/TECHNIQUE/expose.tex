\documentclass[landscape]{seminar}

%\usepackage{french}
\usepackage{graphics}
\usepackage{colordvi}
\usepackage{../EXTENSION/lgrind}
\input{sem-a4.sty}

\title{Techniques de programmation}
\author{Thierry EXCOFFIER}

\begin{document}

\newcommand{\cfile}[1]{\fontsize{7}{8}\selectfont
\input #1
}
\newcommand{\ligne}[0]{\centerline{\rule{10cm}{.1mm}}}

\pagestyle{empty}

\begin{slide}
  \maketitle
\end{slide}


%%%%%%%%%%%%%%%%%%%%%%%%%%%%%%%%%%%%%%%%%%%%%%%%%%%%%%%%%%%%%%%%%%%%%%%%%%%%%%%
\begin{slide}
  \section*{Introduction}

  La suite pr�sente des techniques
  diverses en vari�es de programmation
  utilisables en C.

  Le but est de pallier aux insuffisance
  des langages de programmation.

  La plupart de ces techniques sont
  mises en oeuvre dans Nano�dit.

\end{slide}
%%%%%%%%%%%%%%%%%%%%%%%%%%%%%%%%%%%%%%%%%%%%%%%%%%%%%%%%%%%%%%%%%%%%%%%%%%%%%%%
\begin{slide}
  \section*{Programmation symbolique}

  Le C (et C++) ne permet pas d'acc�der :
  \begin{itemize}
  \item ne permet pas d'acc�der � la liste des champs d'une structure.
  \item ne permet pas d'acc�der au nom de structures, de fonction, d'�num�ration.
  \item ne permet pas de faire des calculs symboliques.
  \item de g�n�rer des fonctions et structures par programmes.
  \end{itemize}

  Plusieurs solutions sont possibles.
\end{slide}

%%%%%%%%%%%%%%%%%%%%%%%%%%%%%%%%%%%%%%%%%%%%%%%%%%%%%%%%%%%%%%%%%%%%%%%%%%%%%%%
\begin{slide}
  \subsection*{Analyseur}

  On peut faire un autre programme qui
  analyse le programme C afin de g�n�rer
  ce qui manque.

  Solution facile mais peut �l�gante car l'on
  a besoin de fichiers supl�mentaires
  et d'�crire l'analyseur.
\end{slide}
%%%%%%%%%%%%%%%%%%%%%%%%%%%%%%%%%%%%%%%%%%%%%%%%%%%%%%%%%%%%%%%%%%%%%%%%%%%%%%%
\begin{slide}
  \subsection*{Pr�processeur}

  Il permet de faire beaucoup de chose,
  malheureusement, il ne permet pas de faire
  de conditions, de tests et de recursions.

  Il est n�anmoins tr�s pratique et permet
  d'�viter beaucoup de redondance dans les
  programmes.

\end{slide}
%%%%%%%%%%%%%%%%%%%%%%%%%%%%%%%%%%%%%%%%%%%%%%%%%%%%%%%%%%%%%%%%%%%%%%%%%%%%%%%
\begin{slide}
  \subsubsection*{Boucle FOR}

Pour �viter la redondance de l'indice de boucle.

\ligne

\cfile{C/for.ctex}
\ligne

\cfile{C/for.cpp}


\end{slide}
%%%%%%%%%%%%%%%%%%%%%%%%%%%%%%%%%%%%%%%%%%%%%%%%%%%%%%%%%%%%%%%%%%%%%%%%%%%%%%%

\begin{slide}
  \subsubsection*{D�buggage}

Il est encore mieux d'avoir la fonction {\tt eprintf}
au lieu de {\tt fprintf}

\ligne

\cfile{C/bug.ctex}
\ligne

\cfile{C/bug.cpp}





\end{slide}
%%%%%%%%%%%%%%%%%%%%%%%%%%%%%%%%%%%%%%%%%%%%%%%%%%%%%%%%%%%%%%%%%%%%%%%%%%%%%%%
\begin{slide}
  \section*{}

\end{slide}
%%%%%%%%%%%%%%%%%%%%%%%%%%%%%%%%%%%%%%%%%%%%%%%%%%%%%%%%%%%%%%%%%%%%%%%%%%%%%%%
\begin{slide}
  \section*{}

\end{slide}
%%%%%%%%%%%%%%%%%%%%%%%%%%%%%%%%%%%%%%%%%%%%%%%%%%%%%%%%%%%%%%%%%%%%%%%%%%%%%%%
\begin{slide}
  \section*{}

\end{slide}
%%%%%%%%%%%%%%%%%%%%%%%%%%%%%%%%%%%%%%%%%%%%%%%%%%%%%%%%%%%%%%%%%%%%%%%%%%%%%%%
\begin{slide}
  \section*{}

\end{slide}
%%%%%%%%%%%%%%%%%%%%%%%%%%%%%%%%%%%%%%%%%%%%%%%%%%%%%%%%%%%%%%%%%%%%%%%%%%%%%%%
\begin{slide}
  \section*{}

\end{slide}
%%%%%%%%%%%%%%%%%%%%%%%%%%%%%%%%%%%%%%%%%%%%%%%%%%%%%%%%%%%%%%%%%%%%%%%%%%%%%%%
\begin{slide}
  \section*{}

\end{slide}
%%%%%%%%%%%%%%%%%%%%%%%%%%%%%%%%%%%%%%%%%%%%%%%%%%%%%%%%%%%%%%%%%%%%%%%%%%%%%%%
\begin{slide}
  \section*{}

\end{slide}
%%%%%%%%%%%%%%%%%%%%%%%%%%%%%%%%%%%%%%%%%%%%%%%%%%%%%%%%%%%%%%%%%%%%%%%%%%%%%%%

\end{document}

